形式一:

$$
    f(n) = \sum_{i=m}^n {n \choose i} \iff g(i) g(n) = \sum_{i=m}^n (-1)^{n-i}{n \choose i} f(i)
$$

形式二: (常用)

$$
    {f(n) = \sum_{i=n}^m {i \choose n} g(i)} \iff {g(n) = \sum_{i=n}^m (-1)^{i-n} {i \choose n} f(i)}
$$

\textbf{常见用法}

钦定 (至少) $k$ 个与恰好 $k$ 个之间的转化

记号:

记 $f(n)$ 表示先钦定至少选 $n$ 个, 再统计钦定情况如此的方案数之和, 其中会包含重复的方案数.

记 $g(n)$ 表示恰好选 $n$ 个的方案数, 不会重复.

那么, 对于 $i \geq n$, $g(i)$ 在 $f(n)$ 中被重复计算了 ${i \choose n}$ 次, 故 $f(n) = \sum_{i=n}^m {i \choose n} g(i)$, 其中 $m$ 为数目上限

通常, $f$ 较易求, 再通过反演即可求 $g$