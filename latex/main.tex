\documentclass[a4paper, twoside]{article}
\title{Standard Code Library}
\author{AntiLeaf}
\date{}

% green= '#10BC72' # '#C3E88D'
% cyan= '#00A1D6' # '#89DDFF'

% \usepackage{zxjatype}
% \setjamainfont{ipaexm.ttf}

\usepackage{graphicx, amssymb, amsmath, textcomp, booktabs}
% \usepackage[libertine, vvarbb]{newtxmath}
\usepackage[scr=rsfso]{mathalfa}
%\usepackage[lining, semibold, type1]{libertine} % a bit lighter than Times--no osf in math
\usepackage[T1]{fontenc} % best for Western European languages
\usepackage{minted}
\usepackage{listings, color, setspace, titlesec, fancyhdr, mdframed, multicol}
\usepackage{fontspec}
\usepackage{ucharclasses}
\usepackage{xunicode, xltxtra}
\usepackage{pdfpages}
\usepackage{tocloft}
\usepackage{nameref}
\usepackage{verbatim}
\usepackage{relsize}
\usepackage[normalem]{ulem}

\usepackage{color,xcolor}

\usepackage{enumitem}
\setenumerate[1]{itemsep=5pt,partopsep=0pt,parsep=\parskip,topsep=5pt,itemindent=1em,leftmargin=3pt}
\setitemize[1]{itemsep=0pt,partopsep=0pt,parsep=\parskip,topsep=5pt,itemindent=1em,leftmargin=3pt}
\setdescription{itemsep=0pt,partopsep=0pt,parsep=\parskip,topsep=5pt,itemindent=1em,leftmargin=3pt}

\setlength{\itemindent}{0em}
\setlength\parindent{0em}

\definecolor{light-gray}{gray}{0.9}    % 1.灰度
\definecolor{black}{gray}{0.0}

\XeTeXlinebreaklocale "zh"
\XeTeXlinebreakskip = 0pt plus 1pt

%configure space between the two columns
\setlength{\columnsep}{13pt}

%configure fonts
\setmonofont{Consolas}[Scale=0.8]
\newfontfamily\substitutefont{等线}[Scale=0.9]
\setTransitionsForChinese{\begingroup\substitutefont}{\endgroup}

%configure minted to display codes
\definecolor{Gray}{rgb}{0.9,0.9,0.9}

%configure section style of table of content
\renewcommand\cftsecfont{\Large}

%configure section style
\titleformat{\section}
{\huge\bfseries}			% The style of the section title
{\thesection }				% a prefix
{15pt}						% How much space exists between the prefix and the title
{}					% How the section is represented
% \titleformat{\section}{\huge}{}{0pt}{}
% \titlespacing{\section}{0pt}{0pt}{0pt}
% \titlespacing{\subsection}{0pt}{0pt}{0pt}
% \titlespacing{\subsubsection}{0pt}{0pt}{0pt}

\usepackage{fancyhdr}
\usepackage[inner=0.9cm, outer=0.7cm, top=1.7cm, bottom=0.4cm]{geometry}
% inner = 1.35cm, outer = 0.9cm

\setlength{\headsep}{0.1cm}
\setlength{\footskip}{0.7cm}

\fancypagestyle{fancy} {

	% \pagestyle{fancy}

	\chead{Standard Code Library}
	\lhead{\nouppercase\leftmark}
	\rhead{\nouppercase\rightmark}
	\cfoot{\thepage}

}

\renewcommand{\headrulewidth}{0.5pt}
\renewcommand{\footrulewidth}{0.5pt}

\setminted[cpp]{
	style=materiallight,
	mathescape,
	linenos,
	autogobble,
	baselinestretch=0.9,
	tabsize=4,
	fontsize=\normalsize,
	%bgcolor=Gray,
	frame=single,
	framesep=1mm,
	framerule=0.3pt,
	numbersep=1mm,
	breaklines=true,
	breaksymbolsepleft=2pt,
	%breaksymbolleft=\raisebox{0.8ex}{ \small\reflectbox{\carriagereturn}}, %not moe!
	%breaksymbolright=\small\carriagereturn,
	breakbytoken=false,
	showtabs=true,
	tab={\relscale{1.08} $\color{light-gray}{\vert} \ \ \ $ \relscale{1}},
}

\setminted[python]{
	style=materiallight,
	mathescape,
	linenos,
	autogobble,
	baselinestretch=0.9,
	tabsize=4,
	fontsize=\normalsize,
	%bgcolor=Gray,
	frame=single,
	framesep=0.8mm,
	framerule=0.3pt,
	numbersep=0.8mm,
	breaklines=true,
	breaksymbolsepleft=2pt,
	%breaksymbolleft=\raisebox{0.8ex}{ \small\reflectbox{\carriagereturn}}, %not moe!
	%breaksymbolright=\small\carriagereturn,
	breakbytoken=false,
	showtabs=true,
	tab={\relscale{1.08} $\color{light-gray}{\vert} \ \ \ $ \relscale{1}},
}

\begin{document}

	\pagestyle{plain}

	\pagenumbering{roman}
	\setcounter{page}{1}

	\begin{multicols}{2}
		
		\begin{spacing}{1}
			\renewcommand{\contentsname}{\huge{目录}}
			\tableofcontents
		\end{spacing}

	\end{multicols}

	\newpage

	\pagestyle{fancy}

	\pagenumbering{arabic}
	\setcounter{page}{1}

	\begin{multicols}{2}

		\section{数学}
			\subsection{FWT}
				矩阵表示:

or 形式 (子集卷积):
$$\begin{aligned}T_{ij} = [i|j=i] = [j \in i]\end{aligned}$$
$$
\begin{aligned}
\begin{pmatrix}
    1&1\\0&1
\end{pmatrix}
\end{aligned}
$$

and 形式 (超集卷积):
$$\begin{aligned}T_{ij} = [i\&j=i] = [i \in j]\end{aligned}$$
$$
\begin{aligned}
\begin{pmatrix}
    1&0\\1&1
\end{pmatrix}
\end{aligned}
$$

xor 形式 ($T$与自身互为逆矩阵):
$$\begin{aligned}T_{ij} = (-1)^{parity(i\&j)}\end{aligned}$$
$$
\begin{aligned}
\begin{pmatrix}
    1&-1\\-1&1
\end{pmatrix}
\end{aligned}
$$

\inputminted{cpp}{../src-midori/math/FWT.cpp}







			\subsection{多项式}
				\subsubsection{FFT}
					\inputminted{cpp}{../src-midori/math/FFT.cpp}
				\subsubsection{NTT}
					\inputminted{cpp}{../src-midori/math/poly.cpp}
				\subsubsection{拉格朗日插值}
					$$\begin{aligned}f(x) = \sum_i f\left(x_i\right)\prod_{j\ne i}\frac{x-x_j}{x_i-x_j}\end{aligned}$$
				\subsubsection{连通图计数}
					设大小为$n$的满足一个限制$P$的简单无向图数量为$g_n$, 满足限制$P$且连通的简单无向图数量为$f_n$, 如果已知$g_{1\dots n}$求$f_n$, 可以得到递推式

$$\begin{aligned}f_n=g_n-\sum_{k=1}^{n-1}{n-1\choose k-1}f_k g_{n-k}\end{aligned}$$

这个递推式的意义就是用任意图的数量减掉不连通的数量, 而不连通的数量可以通过枚举$1$号点所在连通块大小来计算.

注意, 由于$f_0=0$, 因此递推式的枚举下界取$0$和$1$都是可以的.

推一推式子会发现得到一个多项式求逆, 再仔细看看, 其实就是一个多项式$\ln$.
			
			\subsection{线性代数}
				\subsubsection{高斯消元}
					\textbf{辗转相除高斯消元}

对于定义在环上而非域上的矩阵, 利用辗转相除进行消元

下面例子中 mod 未必为素数, 通过辗转相除消元求行列式

\begin{minted}{cpp}
using ll = long long;

int gauss(ll a[][N], int n) {
    int ans = 1;
    for (int j = 0; j < n; ++j) {
        for (int i = j + 1; i < n; ++i) {
            while (a[i][j]) {
                int t = a[j][j] / a[i][j];
                for (int k = j; k < m; ++k)
                    a[j][k] = (a[j][k] + (mod - t) * a[i][k]) % mod; 
                swap(a[i], a[j]);
                ans = mod - ans;
            }
        }
    }
    for (int i = 0; i < n; ++i) 
        ans = ans * a[i][i] % mod;
    return ans;
}
\end{minted}

\textbf{求秩与解线性方程组}

分别维护行列指针, 维护每一列所对应的有效行, 有效行的总数即为秩

线性方程组有唯一解当且仅当 列满秩 且 系数矩阵的秩 等于 增广矩阵的秩

\begin{minted}{cpp}
// 消增广矩阵, 注意第 m 行存放目标向量 
// 若无解,返回空 vector
// 若有无穷多解, 则非主元置 0

vector<ll> gauss(ll a[][N], int n, int m) {
    vector<int> row(m, -1);  // 每个变元所对应的有效行
    vector<ll> ans(m, 0);

    int r = 0;
    for (int c = 0; c < m; ++c) { // 扫描每一列, 用 r 维护
        int sig = -1;
        for (int i = r; i < n; ++i) 
            if(a[i][c]) {
                sig = i; break;
            }
        if (sig == -1) continue; // 无效列

        row[c] = r;
        if (sig != r) 
            swap(a[sig], a[r]);

        ll inv = fpow(a[r][c], mod - 2);
        for(int i = 0; i < n; ++i) {
            if (i == r) continue;
            ll del = inv * a[i][c] % mod;
            for (int j = c; j <= m; ++j) 
                a[i][j] = (a[i][j] + (mod - del) * a[r][j]) % mod; 
        }
        ++r;
    }
    
    if (r < n && a[r][m]) {
        cerr << "no solution!" << endl;
        return {};
    }

    for (int i = 0; i < m; ++i) { // ax = b
        if (row[i] != -1) {
            ans[i] = fpow(a[row[i]][i]) * a[row[i]][m] % mod;
        } else {
            ans[i] = 0; // 非主元置 0 即为合法解
        }
    }

    return ans;
}
\end{minted}

\textbf{矩阵求逆}

维护一个矩阵 $B$, 初始设为 $n$ 阶单位矩阵, 在消元的同时对 $B$ 进行一样的操作, 当把 $A$ 消成单位矩阵时 $B$ 就是逆矩阵.

				\subsubsection{矩阵树定理}
					\textbf{无向图}: 设图$G$的基尔霍夫矩阵$L(G)$等于度数矩阵减去邻接矩阵, 则$G$的生成树个数等于$L(G)$的任意一个代数余子式的值.

\textbf{有向图}: 类似地定义$L_{in}(G)$等于\textbf{入度}矩阵减去邻接矩阵($i$指向$j$有边, 则$A_{i, j} = 1$), $L_{out}(G)$等于\textbf{出度}矩阵减去邻接矩阵.

则以$i$为根的内向树个数即为$L_{out}$的第$i$个主子式(即关于第$i$行第$i$列的余子式), 外向树个数即为$L_{in}$的第$i$个主子式.

(可以看出, 只有无向图才满足$L(G)$的所有代数余子式都相等.)

\textbf{BEST定理(有向图欧拉回路计数)}: 如果$G$是有向欧拉图, 则$G$的欧拉回路的个数等于以一个任意点为根的内/外向树个数乘以$\prod_v (\deg(v) - 1) !$.

并且在欧拉图里, 无论以哪个结点为根, 也无论内向树还是外向树, 个数都是一样的.

另外无向图欧拉回路计数是NP问题.
				\subsubsection{线性基}
					\inputminted{cpp}{../src-midori/math/线性基上二分.cpp}
			
			\subsection{反演与容斥}
				\subsubsection{二项式反演}
					形式一:

$$
    f(n) = \sum_{i=m}^n {n \choose i} \iff g(i) g(n) = \sum_{i=m}^n (-1)^{n-i}{n \choose i} f(i)
$$

形式二: (常用)

$$
    {f(n) = \sum_{i=n}^m {i \choose n} g(i)} \iff {g(n) = \sum_{i=n}^m (-1)^{i-n} {i \choose n} f(i)}
$$

\textbf{常见用法}

钦定 (至少) $k$ 个与恰好 $k$ 个之间的转化

记号:

记 $f(n)$ 表示先钦定至少选 $n$ 个, 再统计钦定情况如此的方案数之和, 其中会包含重复的方案数.

记 $g(n)$ 表示恰好选 $n$ 个的方案数, 不会重复.

那么, 对于 $i \geq n$, $g(i)$ 在 $f(n)$ 中被重复计算了 ${i \choose n}$ 次, 故 $f(n) = \sum_{i=n}^m {i \choose n} g(i)$, 其中 $m$ 为数目上限

通常, $f$ 较易求, 再通过反演即可求 $g$
				\subsubsection{单位根反演}
					$$ [n|k] = \frac{1}{n}\sum_{i=0}^{n-1}\omega_n^{ik} $$

$$ \sum_{i \ge 0} [x^{ik}]f(x) = \frac{1}{k}\sum_{j=0}^{k-1}f(\omega_{k}^j) $$
				\subsubsection{Min-Max容斥}
					设有数集 $S$, 有:

$$
\max(S)=\sum\limits_{\varnothing\ne T\subseteq S} (-1)^{|T|-1} \min(T)
$$

$$
\min(S)=\sum\limits_{\varnothing\ne T\subseteq S} (-1)^{|T|-1} \max(T)
$$

\textbf{第 k 大数的 Min-Max 容斥}

$$
\operatorname{max}_k(S)=\sum\limits_{T\subseteq S,|T|\ge k}(-1)^{|T|-k}\times\binom {|T|-1}{k-1}\times \min(T)
$$

			
			\subsection{组合数}
				\subsubsection{常用组合数}
					\textbf{卡特兰数}

$$C_n = \frac 1 {n + 1}{2n\choose n} = {2n \choose n} - {2n \choose n - 1}$$

\begin{itemize}
	\item $n$个元素按顺序入栈, 出栈序列方案数
	\item 长为$2n$的合法括号序列数
	\item $n + 1$个叶子的满二叉树个数
\end{itemize}

递推式: 
$$C_n = \sum_{i = 0} ^ {n - 1} C_i C_{n - i - 1}$$
$$C_n = C_{n - 1} \frac {4n - 2} {n + 1}$$

普通生成函数:
$$C(x) = \frac {1 - \sqrt {1 - 4 x}} {2 x}$$

扩展: 如果有$n$个左括号和$m$个右括号, 方案数为
$${n + m \choose n} - {n + m \choose m - 1}$$

\textbf{施罗德数}

$$ S_n = S_{n-1} + \sum_{i = 0} ^ {n - 1} S_i S_{n - i - 1} $$
$$ (n + 1)s_n = (6n - 3)s_{n - 1} - (n - 2) s_{n - 2} $$

其中$S_n$是(大)施罗德数, $s_n$是小施罗德数(也叫超级卡特兰数).

除了$S_0 = s_0 = 1$以外, 都有$S_i = 2s_i$.

施罗德数的组合意义:
\begin{itemize}
	\item 从$(0, 0)$走到$(n, n)$, 每次可以走右, 上, 或者右上一步, 并且不能超过$y=x$这条线的方案数
	\item 长为$n$的括号序列, 每个位置也可以为空, 并且括号对数和空位置数加起来等于$n$的方案数
	\item 凸$n$边形的\textbf{任意}剖分方案数
\end{itemize}

(有些人会把大(而不是小)施罗德数叫做超级卡特兰数.)

\textbf{默慈金数}

$$ M_{n + 1} = M_n + \sum_{i = 0} ^ {n - 1} M_i M_{n - 1 - i} = \frac {(2n + 3)M_n + 3n M_{n - 1}} {n + 3} $$

$$ M_n = \sum_{i = 0} ^ {\frac n 2} {n \choose 2i} C_i $$

在圆上的$n$个\textbf{不同的}点之间画任意条不相交(包括端点)的弦的方案数.

也等价于在网格图上, 每次可以走右上, 右下, 正右方一步, 且不能走到$y<0$的位置, 在此前提下从$(0, 0)$走到$(n, 0)$的方案数.

扩展: 默慈金数画的弦不可以共享端点. 如果可以共享端点的话是A054726, 后面的表里可以查到.
				\subsubsection{斯特林数}
					\begin{enumerate}

    \item \textbf{第一类斯特林数}
    
    $n\brack k$表示$n$个元素划分成$k$个轮换的方案数.
    
    递推式: ${n \brack k} = {n-1 \brack k-1} + (n-1){n-1 \brack k}$.
    
    求同一行: 分治FFT $O(n\log ^2 n)$, 或者倍增$O(n\log n)$(每次都是$f(x) = g(x) g(x + d)$的形式, 可以用$g(x)$反转之后做一个卷积求出后者).
    
    $$ \begin{aligned} \sum_{k = 0} ^ n {n \brack k} x^k = \prod_{i = 0} ^ {n - 1} (x + i) \end{aligned} $$
    
    求同一列: 用一个轮换的指数生成函数做$k$次幂
    $$\begin{aligned} \sum_{n = 0} ^ \infty {n \brack k} \frac {x ^ n} {n!} = \frac {\left(\ln (1 - x)\right) ^ k} {k!} = \frac {x ^ k} {k!} \left( \frac {\ln (1 - x)} x \right) ^ k \end{aligned}$$
    
    \item \textbf{第二类斯特林数}
    
    $n\brace k$表示$n$个元素划分成$k$个子集的方案数.
    
    递推式: ${n \brace k} = {n-1 \brace k-1} + k{n-1 \brace k}$.
    
    求一个: 容斥, 狗都会做
    
    $$\begin{aligned} {n \brace k} = \frac 1 {k!} \sum_{i = 0} ^ k (-1) ^ i {k \choose i} (k - i) ^ n = \sum_{i = 0} ^ k \frac {(-1) ^ i} {i!} \frac {(k - i) ^ n} {(k - i)!} \end{aligned}$$
    
    求同一行: FFT, 狗都会做
    
    求同一列: 指数生成函数
    
    $$\begin{aligned} \sum_{n = 0} ^ \infty {n \brace k} \frac {x ^ n} {n!} = \frac {\left(e ^ x - 1\right) ^ k} {k!} = \frac {x ^ k} {k!} \left( \frac {e ^ x - 1} x \right) ^ k \end{aligned}$$
    
    普通生成函数
    
    $$\begin{aligned} \sum_{n = 0} ^ \infty {n \brace k} x ^ n = x ^ k \left(\prod_{i = 1} ^ k (1 - i x)\right) ^ {-1} \end{aligned}$$
    
    \item \textbf{幂的转换}
    
    \textbf{上升幂与普通幂的转换}
    
    $$ x^{\overline{n}}=\sum_{k} {n \brack k} x^k $$
    
    $$ x^n=\sum_{k} {n \brace k} (-1)^{n-k} x^{\overline{k}} $$
    
    \textbf{下降幂与普通幂的转换}
    
    $$ x^n=\sum_{k} {n \brace k} x^{\underline{k}} = \sum_{k} {x \choose k} {n \brace k} k! $$
    
    $$ x^{\underline{n}}=\sum_{k} {n \brack k} (-1)^{n-k} x^k $$
    
    另外, 多项式的\textbf{点值}表示的每项除以阶乘之后卷上$e^{-x}$乘上阶乘之后是牛顿插值表示, 或者不乘阶乘就是\textbf{下降幂}系数表示. 反过来的转换当然卷上$e^x$就行了. 原理是每次差分等价于乘以$(1 - x)$, 展开之后用一次卷积取代多次差分.
    
    \item \textbf{斯特林多项式(斯特林数关于斜线的性质)}
    
    定义:
    
    $$ \sigma_n(x) = \frac {{x\brack n}} {x(x-1)\dots(x-n)} $$
    
    $\sigma_n(x)$的最高次数是$x^{n - 1}$. (所以作为唯一的特例, $\sigma_0(x) = \frac 1 x$不是多项式.)
    
    斯特林多项式实际上非常神奇, 它与两类斯特林数都有关系.
    
    $$ {n \brack n-k} = n^{\underline{k+1}} \sigma_k(n) $$
    
    $$ {n \brace n-k} = (-1)^{k+1} n^{\underline{k+1}} \sigma_k(-(n-k)) $$
    
    不过它并不好求. 可以$O(k^2)$直接计算前几个点值然后插值, 或者如果要推式子的话可以用后面提到的二阶欧拉数.
    
    \end{enumerate}
				\subsubsection{欧拉数}
					\def \bangle{ \atopwithdelims \langle \rangle}

\begin{enumerate}

\item \textbf{欧拉数}

${n\bangle k}$: $n$个数的排列, 有$k$个上升的方案数.

$$ {n\bangle k} = (n - k){n - 1 \bangle k - 1} + (k + 1){n - 1 \bangle k} $$

$$ {n\bangle k} = \sum_{i = 0} ^ {k + 1} (-1)^i {n + 1\choose i} (k + 1 - i)^n $$

$$ \sum_{k = 0} ^ {n - 1} {n\bangle k} = n! $$

$$ x^n = \sum_{k = 0} ^ {n - 1} {n\bangle k} {x+k \choose n} $$

$$ k!{n\brace k} = \sum_{i = 0} ^ {n - 1} {n\bangle i} {i\choose n - k} $$

\item \textbf{二阶欧拉数}

$\left\langle\!\!{n\bangle k}\!\!\right\rangle$: 每个数都出现两次的多重排列, 并且每个数两次出现之间的数都比它要大. 在此前提下有$k$个上升的方案数.

$$ \left\langle\!\!{n\bangle k}\!\!\right\rangle = (2n-k-1)\left\langle\!\!{n-1\bangle k-1}\!\!\right\rangle + (k+1)\left\langle\!\!{n-1 \bangle k}\!\!\right\rangle $$

$$ \sum_{k = 0} ^ {n - 1} \left\langle\!\!{n\bangle k}\!\!\right\rangle = (2n-1)!! = \frac{(2n)^{\underline n}} {2^n} $$

\item \textbf{二阶欧拉数与斯特林数的关系}

$$ {x \brace x-n} = \sum_{k = 0} ^ {n - 1} \left\langle\!\!{n\bangle k}\!\!\right\rangle {x + n - k - 1 \choose 2n} $$

$$ {x \brack x-n} = \sum_{k = 0} ^ {n - 1} \left\langle\!\!{n\bangle k}\!\!\right\rangle {x + k \choose 2n} $$

\end{enumerate}

			\subsection{线性规划}
				\subsubsection{对偶原理}
					给定一个原始线性规划:

$$
\begin{aligned}
\text{Minimize}&&\sum_{j=1}^n c_j x_j\\
\text{Subject to}&&\sum_{j=1}^n a_{ij} x_j\ge b_i,\\
&&x_j\ge 0
\end{aligned}
$$

定义它的对偶线性规划为:

$$
\begin{aligned}
\text{Maximize}&&\sum_{i=1}^m b_i y_i\\
\text{Subject to}&&\sum_{i=1}^m a_{ij} y_i\le c_j,\\
&&y_i\ge 0
\end{aligned}
$$

用矩阵可以更形象地表示为:
$$
\begin{aligned}
\text{Minimize}&& \mathbf c^T \mathbf x &&&& \text{Maximize} && \mathbf b^{T}\mathbf y\\
\text{Subject to}&& A\mathbf x \ge \mathbf b, && \Longleftrightarrow && \text{Subject to} && A^T\mathbf y \le \mathbf c,\\
&& \mathbf x\ge 0 &&&&&& \mathbf y\ge 0
\end{aligned}
$$
				
			\subsection{杂项}
				\subsubsection{约瑟夫环}
					\inputminted{cpp}{../src-midori/math/约瑟夫环.cpp}
				\subsubsection{杨辉三角行区间和}
					\inputminted{cpp}{../src-midori/math/杨辉三角行区间和.cpp}
				\subsubsection{辛普森积分}
					\inputminted{cpp}{../src-midori/math/simpson.cpp}
				\subsubsection{纳什均衡}
					首先定义纯策略和混合策略: 纯策略是指你一定会选择某个选项, 混合策略是指你对每个选项都有一个概率分布$p_i$, 你会以相应的概率选择这个选项.

考虑这样的游戏: 有几个人(当然也可以是两个)各自独立地做决定, 然后同时公布每个人的决定, 而每个人的收益和所有人的选择有关.

那么纳什均衡就是每个人都决定一个混合策略, 使得在其他人都是纯策略的情况下, 这个人最坏情况下(也就是说其他人的纯策略最针对他的时候)的收益是最大的. 也就是说, 收益函数对这个人的混合策略求一个偏导, 结果是0(因为是极大值).

纳什均衡点可能存在多个, 不过在一个双人\textbf{零和}游戏中, 纳什均衡点一定唯一存在.
				\subsubsection{康托展开}
					求排列的排名: 先对每个数都求出它后面有几个数比它小(可以用树状数组预处理), 记为$c_i$, 则排列的排名就是

$$ \sum_{i = 1} ^ n c_i (n - i)! $$

已知排名构造排列: 从前到后先分别求出$c_i$, 有了$c_i$之后再用一个平衡树(需要维护排名)倒序处理即可.
				\subsubsection{常用NTT素数及原根}
					\begin{tabular}{|c|c|c|c|}
	\hline $p = r \times 2 ^ k + 1$ &  $r$  & $k$  & 最小原根 \\
	\hline $104857601$              &  $25$ & $22$ & $3$ \\
	\hline $167772161$              &  $5$  & $25$ & $3$ \\
     \hline $469762049$      &  $7$   &  $26$  &    $3$ \\
     \hline $985661441$      & $235$  &  $22$  &    $3$ \\
  \hline $\mathbf{998244353}$    & $119$  &  $23$  &    $3$ \\
  \hline $\mathbf{1004535809}$    & $479$  &  $21$  &    $3$ \\
  \hline $1005060097 ^ *$      & $1917$ &  $19$  &  $\emph{5}$ \\
  \hline $\emph{2013265921}$    &  $15$  &  $27$  &  $\emph{31}$ \\
    \hline $2281701377$      &  $17$  &  $27$  &    $3$ \\
 \hline $31525197391593473$  &  $7$   &  $52$  &    $3$ \\
\hline $180143985094819841$  &  $5$   &  $55$  &  $\emph{6}$ \\
\hline $1945555039024054273$ &  $27$  &  $56$  &  $\emph{5}$ \\
\hline $4179340454199820289$ &  $29$  &  $57$  &    $3$ \\
\hline
\end{tabular}

*注: $1005060097$有点危险, 在变化长度大于$524288 = 2 ^ {19}$时不可用.

		\newpage
		\section{数论}

			\subsection{常见预处理与快速幂}
				\inputminted{cpp}{../src-midori/number/常见预处理与快速幂.cpp}

			\subsection{因数分解与素性判定}
				\subsubsection{朴素因数分解}
					\inputminted{cpp}{../src-midori/number/素因数分解.cpp}
				\subsubsection{Miller-Rabin 与 Pollard-Rho}
					\inputminted{cpp}{../src-midori/number/miller_rabin-pollard_rho.cpp}

			\subsection{筛法}
				\subsubsection{线性筛}
					\inputminted{cpp}{../src-midori/number/线性筛.cpp}
				\subsubsection{Min25 筛}
					\inputminted{cpp}{../src-midori/number/min25.cpp}
				% \subsection{杜教筛}

			\subsection{扩展欧几里得}
				\inputminted{cpp}{../src-midori/number/exgcd.cpp}
			
			\subsection{扩展欧拉定理}
				$$
a^b \equiv a^{b\,\bmod\,{\phi(p)}}, \, (a, b) = 1
$$

$$
a^b \equiv a^{b\,\bmod\,{\phi(p)} + \phi(p)}, \, (a, b) \neq 1
$$

			\subsection{中国剩余定理}
				% \subsection{基本公式}
				\subsubsection{两个数的 crt}
					\inputminted{cpp}{../src-midori/number/两个数的crt.cpp}
				\subsubsection{excrt}
					\inputminted{cpp}{../src-midori/number/excrt.cpp}
			
			\subsection{卢卡斯定理}
				\subsubsection{模素数卢卡斯}
					\inputminted{cpp}{../src-midori/number/lucas.cpp}
				\subsubsection{扩展卢卡斯}
					\inputminted{cpp}{../src-midori/number/exlucas.cpp}
			
			\subsection{原根与离散对数}
				\subsubsection{原根}
					\inputminted{cpp}{../src-midori/number/原根.cpp}
				\subsubsection{BSGS}
					\inputminted{cpp}{../src-midori/number/bsgs.cpp}
			
			% \subsection{二次剩余}

			\subsection{杂项}
				\subsubsection{大数整除小数取模}
					计算 ${a \over b} \bmod p$:

当 a 的本值太大无法表示时,可以计算 a 对 b * p 取模的结果,再除 b,模 p

$$ {a \over b} \bmod p = {{ a \bmod {b * p} } \over b } \bmod p $$
				\subsubsection{立方根复杂度求 mobius 函数}
					\inputminted{cpp}{../src-midori/number/立方根求mu.cpp}
				\subsubsection{直接求 euler 函数}
					\inputminted{cpp}{../src-midori/number/直接求phi.cpp}
		
		\newpage
		\section{数据结构}
			\subsection{点分治}
				\inputminted{cpp}{../src-midori/datastructure/点分治.cpp}
			
			\subsection{树链剖分}
				\inputminted{cpp}{../src-midori/datastructure/树链剖分.cpp}
			
			\subsection{主席树}
				\inputminted{cpp}{../src-midori/datastructure/主席树.cpp}
			
			\subsection{线段树合并}
				\inputminted{cpp}{../src-midori/datastructure/线段树合并.cpp}

			\subsection{zkw 线段树}
				\inputminted{cpp}{../src-midori/datastructure/zkw-segtree.cpp}
			
			\subsection{矩形面积并}
				\inputminted{cpp}{../src-midori/datastructure/矩形面积并.cpp}
				
		\newpage
		\section{图论}

			\subsection{2-sat}
				% \inputminted{cpp}{../src-midori/graph/2-sat.cpp}
			
			\subsection{Dinic 网络流}
				\inputminted{cpp}{../src-midori/graph/dinic.cpp}
			
			\subsection{Dijkstra 费用流}
				\inputminted{cpp}{../src-midori/graph/dijkstra费用流.cpp}
			
			\subsection{二分图最大带权匹配}
				\inputminted{cpp}{../src-midori/graph/二分图最大带权匹配.cpp}
			
			\subsection{带花树}
				\inputminted{cpp}{../src-midori/graph/带花树.cpp}
			
			\subsection{Hall 定理}
				\textbf{霍尔定理}

二分图中, 左侧点集 $X$ 存在最大匹配的充要条件是:

$X$ 中的任意 $k$ 个点至少与 $Y$ 中 $k$ 个点相邻.

推论: 正则二分图存在完美匹配 (正则图 指每个点度数相等的图)
			

		\newpage
		\section{字符串}

			\subsection{后缀自动机}
				\inputminted{cpp}{../src-midori/string/后缀自动机.cpp}
			
			\subsection{广义后缀自动机}
				\subsubsection{对 Trie 建自动机}
					以 bfs 遍历 Trie 上的每一个结点, 以 父结点 在 SAM 中对应的结点为 last 调用 insert 即可.
				\subsubsection{对多模式串建自动机}
					\inputminted{cpp}{../src-midori/string/广义后缀自动机-多模式串.cpp}
				
			\subsection{AC 自动机}
				\inputminted{cpp}{../src-midori/string/AC自动机.cpp}
			
			\subsection{后缀数组}
				\subsubsection{倍增}
					\inputminted{cpp}{../src-midori/string/SA.cpp}

				\subsubsection{SAIS}
					\inputminted{cpp}{../src-midori/string/SAIS.cpp}
			
			\subsection{马拉车}
				\inputminted{cpp}{../src-midori/string/manacher.cpp}
		
		\newpage
		\section{动态规划}
			\subsection{数位 DP}
				\inputminted{cpp}{../src-midori/dp/数位dp.cpp}
		
		\newpage
		\section{几何}
			\inputminted{cpp}{../src-midori/geometry/geo-jry.cpp}
		
		\newpage
		\section{杂项}
			\subsection{java 高精度}
				\inputminted{java}{../src-midori/other/高精度.java}
			\subsection{重载 umap}
				\inputminted{cpp}{../src-midori/other/重载umap.cpp}
			\subsection{bitset}
				\inputminted{cpp}{../src-midori/other/bitset.cpp}
			\subsection{模拟退火}
				\inputminted{cpp}{../src-midori/other/模拟退火.cpp}
			\subsection{对拍}
				\subsubsection{windows 对拍}
					\inputminted{bat}{../src-midori/other/duipai.bat}
				\subsubsection{linux 对拍}
					\inputminted{shell}{../src-midori/other/duipai.sh}
			\subsection{快读}
				\inputminted{cpp}{../src-midori/other/fread.cpp}
			\subsection{随机}
				\inputminted{cpp}{../src-midori/other/rand.cpp}
			\subsection{python}
				\inputminted{python}{../src-midori/python/sort.py}
		
	\end{multicols}

\end{document}